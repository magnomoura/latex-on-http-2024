\documentclass[12pt]{article}
\usepackage[T1]{fontenc}
\usepackage[utf8]{inputenc}
\usepackage{fourier} 
\usepackage[scaled=0.875]{helvet}
\renewcommand{\ttdefault}{lmtt}
\usepackage{amsmath,amssymb,makeidx}
\usepackage{fancybox}
\usepackage{graphicx}
\usepackage{tabularx}
\usepackage[normalem]{ulem}
\usepackage{pifont}
\usepackage{textcomp} 
\usepackage{verbatim}
\newcommand{\euro}{\eurologo}
\usepackage{pgf,tikz}
\usetikzlibrary{arrows}
\usepackage{ulem}
\usepackage{dcolumn}
\usepackage{lscape}
\usepackage{diagbox}
\usepackage{lscape}
\usepackage{pst-plot,pst-text,pst-tree}
\usepackage{pstricks,pst-plot,pst-tree,pst-func,pstricks-add}
\newcommand{\R}{\mathbb{R}}
\newcommand{\N}{\mathbb{N}}
\newcommand{\D}{\mathbb{D}}
\newcommand{\Z}{\mathbb{Z}}
\newcommand{\Q}{\mathbb{Q}}
\newcommand{\C}{\mathbb{C}}


\setlength{\textheight}{23,5cm}
\newcommand{\vect}[1]{\mathchoice%
{\overrightarrow{\displaystyle\mathstrut#1\,\,}}%
{\overrightarrow{\textstyle\mathstrut#1\,\,}}%
{\overrightarrow{\scriptstyle\mathstrut#1\,\,}}%
{\overrightarrow{\scriptscriptstyle\mathstrut#1\,\,}}}
\renewcommand{\theenumi}{\textbf{\arabic{enumi}}}
\renewcommand{\labelenumi}{\textbf{\theenumi.}}
\renewcommand{\theenumii}{\textbf{\alph{enumii}}}
\renewcommand{\labelenumii}{\textbf{\theenumii.}}
\def\Oij{$\left(\text{O},~\vect{\imath},~\vect{\jmath}\right)$}
\def\Oijk{$\left(\text{O},~\vect{\imath},~\vect{\jmath},~\vect{k}\right)$}
\def\Ouv{$\left(\text{O},~\vect{u},~\vect{v}\right)$}
\setlength{\voffset}{-1,5cm}
\usepackage{fancyhdr}
\PassOptionsToPackage{hyphens}{url}   % Pour les liens qui etaient coupes en bas de page
\usepackage{hyperref}

\hypersetup{%
  pdfauthor = {Mireille Coilhac, Vincent-Xavier Jumel},
  %pdfsubject = {controle},
  pdftitle = {Projet : proposer des groupes à partir d'un graphe d'interaction sociales },
  allbordercolors = white,
  pdfstartview=FitH
}

\usepackage[frenchb]{babel}
\usepackage[np]{numprint}
\usepackage[left=1.8cm,right=1.8cm,top=3cm,bottom=1cm]{geometry}
\usepackage { fancybox }
\usepackage{amsmath}
\usepackage{lastpage}

\begin{document}

\setlength\parindent{0mm}
\renewcommand{\headrulewidth}{0pt}%supprime le trait d'entête


\lfoot{\small {M. \bsc{COILHAC}, V.X. \bsc{JUMEL}\\
Ce document est mis à disposition selon les termes de la Licence Creative
Commons Attribution - Pas d’utilisation commerciale - Partage dans les mêmes
conditions 4.0 non transposé }}
%\rfoot{\small{avril 2015}}
\cfoot{}
\rfoot{\thepage/\pageref{LastPage}} %numerotation de page automatique
\renewcommand \footrulewidth{.2pt}

\pagestyle{fancy}
\thispagestyle{empty}

\begin{center} %

\Huge \textbf{ Former des groupes de travail dans votre classe de NSI}
\end{center}

\vspace{1cm}

\section{Situation}\

La classe de première NSI est composée de 27 élèves qui doivent réaliser un projet informatique par groupes de 3.\\
Vous devez réaliser pour le professeur, qui est votre «client», un programme qui fasse une proposition de la composition de ces groupes de 3 élèves, et qui leur convienne le mieux possible .

\medskip

Pour vous aider, le professeur a réalisé une petite enquête : il a demandé à chaque élève de citer:
\begin{itemize}
 \item [\textbullet] un élève avec qui il voudrait absolument travailler (affecté de la valeur +2),
 \item [\textbullet]un avec qui il voudrait travailler (affecté de la valeur +1)
 \item [\textbullet]et un avec qui il ne voudrait pas travailler (affecté de la valeur -1)
\end{itemize}

Ces données vont être fournies par le professeur sous la forme d’un dictionnaire :  
Par exemple : 
\begin{verbatim}
Classe_NSI={ 'Alexia': {'Kellian': 2, 'Christelle': 1, 'Alice': -1}, etc….}  
\end{verbatim}

Ici Alexia voudrait absolument travailler avec Kellian, voudrait bien travailler avec Christelle, et pas du tout avec Alice.\\

\medskip

\section{Cahier des charges }

\begin{itemize}
 
\item [\textbullet] \textbf{Version obligatoire}


\begin{itemize}
 
\item [\textbullet] On ne dispose que des élèves avec qui chacun veut travailler : 
\begin{verbatim}
Par exemple : Classe_NSI={ 'Alexia': {'Kellian': 2, 'Christelle': 1}, etc….}.  
\end{verbatim}
\item [\textbullet] Ce dictionnaire sera dans le fichier data \_exemple.json  
\item [\textbullet]Le professeur fournit la fonction def lire-entree \_json qui retourne le dictionnaire.
\item [\textbullet] On ne tient pas compte des préférences (par exemple celle pour Kellian par rapport à Christelle).  
\end{itemize}
\item [\textbullet] \textbf{Version luxe}\\
Tenir compte de tous les renseignements donnés (+2, +1 ou -1)
\end{itemize}
\vspace{1cm}
\medskip

\setlength\parindent{20mm}
\textbf{Dans tous les cas, vous devrez travailler par groupes de deux élèves:}
\setlength\parindent{0mm}


\begin{enumerate}
    \item Réaliser en Python le programme demandé, avec des tests pour les principales fonction.
    \item Faire un exposé avec présentation, et démonstration de l'exécution du programme.
    \item Rédiger un dossier contenant (au moins 10 pages hors annexe) :
    \begin{itemize}
        \item [\textbullet] le cahier des charges
        \item [\textbullet] un planning des différentes tâches exécutées par chacun
        \item [\textbullet] une explication des problèmes rencontrés
        \item [\textbullet] des copies d'écran des résultats obtenus
        \item [\textbullet] le code complet du programme en annexe
    \end{itemize}
\end{enumerate}
 
       
\section{Planning}

\begin{enumerate}

\item Phase 0 : Mise en place de framagit (0,5 semaine)

\medskip

\item Phase 1 : (2 semaines)
\begin{itemize}

\item [\textbullet] 1\ier{} lot : Représentation des graphes :

Créer un algorithme, qui crée un dictionnaire substituant des numéros aux prénoms, ne gardant que les renseignements nécessaires, (pas de +2 ou +1) pour que les graphes soient plus lisible. \\
Faire une dictionnaire de correspondance numéro/prénom

Écrire une fonction qui crée un graphe orienté à partir d’un dictionnaire

Représenter un graphe donné par un dictionnaire, et créer un fichier png avec ce graphique (utiliser les bibliothèques matplotlib, networkx, numpy…)
\medskip
\item [\textbullet]2\ieme{} lot : Travail sur les chemins : \\
On commence par travailler sans se préoccuper des choix +2 ou +1. \\
Créer des fonctions, qui à partir d’un dictionnaire du type \{0 :[20,15],1 :[18,13], ...\} permettent de déterminer tous les cycles de longueur 3.  
Vous pouvez par exemple commencer par chercher tous les chemins de longueur 3.  


Créer un graphe réduit dans lequel tous les cycles de longueur 3 ont été supprimés.
\end{itemize}
\medskip
\item Phase 2 : (1 semaine)

Réunion des porteurs du projet: mise en commun du travail réalisé et intégration des différentes fonctions. Représenter les graphes réduit créés dans le 2\ieme{} lot.  
Vous réaliserez ce travail à 2, en classe.

Point d'étape avec votre client (le professeur).

\medskip
\item Phase 3 : (2 semaines)
\begin{itemize}


\item [\textbullet] 1\ier{} lot : travail sur la matrice d’adjacence

Créer une fonction qui  à partir d’un dictionnaire du type \{0 :[20,15],1 :[18,13],... \} retourne la matrice d’adjacence.

\item [\textbullet] 2\ieme{} lot :

Créer une fonction qui retourne la liste des élèves qui ne sont demandés par personne

Créer une fonction qui retourne la liste des binômes : Par exemple si Alexia(5) veut travailler avec Kellian(11), et si Kellian(11) veut travailler avec Alexia(5) cette fonction retournera entre autres [5,11]. On propose de donner la liste par ordre croissant.



\medskip
\end{itemize}
\item Phase 4 : (1 semaine)  

Réunion des porteurs du projet: mise en commun, aide mutuelle.

Détermination des lots restant à réaliser pour finaliser le projet.  
Répartition du travail entre vous
Point d'étape avec votre client (le professeur).
\medskip

\item Phase 5 : (2 semaines)

Chacun réalise le lot décidé en phase 4.\\
Point d'étape hebdomadaire avec votre client (le professeur).

\medskip

\item Phase 6 : (2 semaines)  

Mise en commun du travail, intégration des différentes fonctions dans le programme.  \\
Point d'étape hebdomadaire avec votre client (le professeur).\\
Rectifications éventuelles

 \medskip

\item Phase 7 : (1 semaine)  \\
Préparation de l'exposé et de tous les supports demandés (diaporama, dossier...)
\end{enumerate}

\section{Évaluation du projet}\
Les points suivants seront évalués : \\

\begin{itemize}
   \item [\textbullet] Le programme fonctionne et répond au cahier des charges
   \item [\textbullet] engagement personnel : chaque élève évalue de façon confidentielle auprès du professeur l'implication de son camarade dans le projet.
  \item [\textbullet] respect des dates du planning
  \item [\textbullet] qualité du programme créé (structure générale, présence de tests, de commentaires etc...)
  \item [\textbullet] qualité de la prestation orale lors de l'exposé
  \item [\textbullet] qualité des supports (diaporama, dossier)

\end{itemize}
\subsection{Grille d'évaluation}

\begin{tabular}{|p{8cm}|p{5cm}|}
  \hline
  Critère & Niveau (MI, MF, BM, TBM) \\ \hline
  Respect du cahier des charges & \\ \hline
  Le code est correctement exécutable & \\ \hline
  Les tests sont respectés et validés & \\ \hline
  Les fonctions sont documentées & \\ \hline
  Toutes les fonctions sont testées & \\ \hline
  Les problèmes sont identifiés par des tickets & \\ \hline
  Les dates des jalons ont été respectées & \\ \hline
  Avis des camarades sur l'implication & \\ \hline
  Avis de l'enseignant sur l'implication & \\ \hline
\end{tabular}

\section{Quelques ressources documentaires sur le sujet}

\begin{enumerate}
    


\item  Un problème complexe :

  \url{ https://fr.wikipedia.org/wiki/Partitionnement_de_graphe}\\
  \url{ https://en.wikipedia.org/wiki/Graph_partition}\\
\medskip
\item Des solutions algorithmiques :\\
\url{https://en.wikipedia.org/wiki/Cycle_detection}\\
  \url{ https://stackoverflow.com/questions/10185773/graph-theory-splitting-a-graph}\\
  \url{ https://cstheory.stackexchange.com/questions/16387/k-clustering-of-a-graph-maximizing-intra-cluster-weights}
\medskip
\item ou pratiques :\\
   \url{ https://stackoverflow.com/questions/53379575/networkx-splitting-a-graph-into-n-subgraphs-based-on-edge-weight}\\
   %PB adresse suivante à cause des % - A modifier
   \url{ https://framagit.org/ketluts/gestiondeclasses/blob/master/app%2Fstudent%2Fviews%2Fsociogramme.php}\\ 



\end{enumerate}
\end{document}
%src : https://www.overleaf.com/project/5dc5a85e73ab9f00010810a7
